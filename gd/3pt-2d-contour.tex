\documentclass[tikz]{standalone}
\usepackage{pgfplots}
\usepackage{xcolor}
\usepackage{booktabs}
\usepgfplotslibrary{patchplots}
\usetikzlibrary{arrows.meta}
\definecolor{brown}{HTML}{8a4e3d}
\definecolor{blue}{HTML}{6d9eeb}
\definecolor{gray}{HTML}{888888}
\definecolor{gold}{HTML}{ffd966}
\definecolor{green}{HTML}{6aa84f}
\pgfplotsset{compat=1.18}

\begin{document}
\begin{tikzpicture}
  \begin{axis}[
    set layers,
    view={0}{90},
    xlabel={\(\theta_1\)},
    ylabel={\(\theta_2\)},
    xmin=-0.75, xmax=1.5,
    ymin=0, ymax=2.5,
    xtick={-0.5,0,0.5,1,1.5},
    ytick={0,0.5,1,1.5,2,2.5},
    minor x tick num=1,
    minor y tick num=1,
    grid=both,
    major grid style={line width=0.2pt,draw=black},
    minor grid style={line width=0.1pt,draw=gray!50,dashed},
    width=9cm,
    height=10cm,
    scale only axis,
    axis lines=left,
    x axis line style={->,>=stealth, shorten >=-3pt},
    y axis line style={->,>=stealth, shorten >=-3pt},
  ]
    % Data: p1=(1,2,3), p2=(2,1,2), p3=(3,4,6), h=θ₁x₁+θ₂x₂
    % J₁ = (3-θ₁-2θ₂)², J₂ = (2-2θ₁-θ₂)², J₃ = (6-3θ₁-4θ₂)²
    % J = 1/3[J₁+J₂+J₃]
    % Minimum at θ₁≈0.342, θ₂≈1.263
    \addplot3[
      contour gnuplot={
        levels={0.1, 0.25, 0.5, 1, 2, 4, 8, 16},
        draw color=gold!80!black,
        labels=false,
      },
      domain=-0.75:1.5, domain y=0:2.5,
      samples=120,
    ] {((3-x-2*y)^2 + (2-2*x-y)^2 + (6-3*x-4*y)^2)/3};

    % Mark minimum at (13/38, 24/19) ≈ (0.342, 1.263)
    \addplot[only marks, mark=star, mark options={scale=2.5, fill=red}]
      coordinates {(0.342,1.263)};

    % Evaluation point (1, 1)
    % residual₁ = 3-1-2 = 0       → ∇J₁ = [0, 0] (on J₁'s zero!)
    % residual₂ = 2-2-1 = -1      → ∇J₂ = -2(-1)[2,1] = [4, 2]
    % residual₃ = 6-3-4 = -1      → ∇J₃ = -2(-1)[3,4] = [6, 8]
    % ∇J = 1/3[0+4+6, 0+2+8] = [10/3, 10/3]
    \addplot[only marks, mark=diamond*, mark options={scale=3, fill=gold!80!black}]
      coordinates {(1,1)};

    % Gradient arrows (scaled by 0.06)
    % ∇J₁ = [0,0] — zero gradient, no arrow

    % ∇J₂ = [4,2] → [0.24, 0.12]
    \draw[-{Stealth[length=6pt,width=4pt]}, line width=2pt, blue]
      (axis cs:1,1) -- (axis cs:1.24,1.12);

    % ∇J₃ = [6,8] → [0.36, 0.48]
    \draw[-{Stealth[length=6pt,width=4pt]}, line width=2pt, gray]
      (axis cs:1,1) -- (axis cs:1.36,1.48);

    % ∇J = [10/3, 10/3] → [0.2, 0.2]
    \draw[-{Stealth[length=6pt,width=4pt]}, line width=2pt, green!80!black]
      (axis cs:1,1) -- (axis cs:1.2,1.2);

    % Gradient legend (on foreground layer, above grid)
    \begin{pgfonlayer}{axis foreground}
    \node[anchor=north east, fill=white, draw=gray!50, inner sep=6pt, rounded corners=2pt]
      at (axis cs:1.45,2.45) {%
      \small
      \begin{tabular}{@{}cl@{}}
        \tikz[baseline=-0.3ex]\draw[-{Stealth[length=4pt,width=3pt]}, line width=2pt, brown] (0,0) -- (0.18,0); & $\nabla J_1 = 0$ \\
        \tikz[baseline=-0.3ex]\draw[-{Stealth[length=4pt,width=3pt]}, line width=2pt, blue] (0,0) -- (0.18,0); & $\nabla J_2$ \\
        \tikz[baseline=-0.3ex]\draw[-{Stealth[length=4pt,width=3pt]}, line width=2pt, gray] (0,0) -- (0.18,0); & $\nabla J_3$ \\
        \tikz[baseline=-0.3ex]\draw[-{Stealth[length=4pt,width=3pt]}, line width=2pt, green!80!black] (0,0) -- (0.18,0); & $\nabla J$ \\
      \end{tabular}%
    };

    % Dataset table
    \node[anchor=north west, fill=white, draw=gray!50,
          inner sep=6pt, rounded corners=2pt] at (axis cs:-0.7,2.45) {%
      \small
      \begin{tabular}{@{}c ccc@{}}
        \toprule
        & $x_1$ & $x_2$ & $y$ \\
        \midrule
        \textcolor{brown}{$p_1$} & 1 & 2 & 3 \\
        \textcolor{blue}{$p_2$} & 2 & 1 & 2 \\
        \textcolor{gray}{$p_3$} & 3 & 4 & 6 \\
        \bottomrule
      \end{tabular}%
    };
    \end{pgfonlayer}
  \end{axis}
\end{tikzpicture}
\end{document}
