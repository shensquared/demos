\documentclass[tikz]{standalone}
\usepackage{pgfplots}
\usepackage{xcolor}
\usepackage{booktabs}
\usepgfplotslibrary{patchplots}
\usetikzlibrary{arrows.meta}
\definecolor{brown}{HTML}{8a4e3d}
\definecolor{blue}{HTML}{6d9eeb}
\definecolor{gray}{HTML}{888888}
\definecolor{gold}{HTML}{ffd966}
\definecolor{green}{HTML}{6aa84f}
\definecolor{hpurple}{HTML}{674ea7}
\pgfplotsset{compat=1.18}

\begin{document}
\begin{tikzpicture}
  \begin{axis}[
    set layers,
    view={0}{90},
    xlabel={\(\theta_1\)},
    ylabel={\(\theta_2\)},
    xmin=-0.75, xmax=1.5,
    ymin=0, ymax=2.5,
    xtick={-0.5,0,0.5,1,1.5},
    ytick={0,0.5,1,1.5,2,2.5},
    minor x tick num=1,
    minor y tick num=1,
    grid=both,
    major grid style={line width=0.2pt,draw=black},
    minor grid style={line width=0.1pt,draw=gray!50,dashed},
    width=9cm,
    height=10cm,
    scale only axis,
    axis lines=left,
    x axis line style={->,>=stealth, shorten >=-3pt},
    y axis line style={->,>=stealth, shorten >=-3pt},
  ]
    % Data: p1=(1,2,3), p2=(2,1,2), p3=(3,4,6), h=θ₁x₁+θ₂x₂
    % J₁ = (3-θ₁-2θ₂)², J₂ = (2-2θ₁-θ₂)², J₃ = (6-3θ₁-4θ₂)²
    % J = 1/3[J₁+J₂+J₃]
    % Minimum at θ₁≈0.342, θ₂≈1.263
    \addplot3[
      contour gnuplot={
        levels={0.1, 0.25, 0.5, 1, 2, 4, 8, 16},
        draw color=gold!80!black,
        labels=false,
      },
      domain=-0.75:1.5, domain y=0:2.5,
      samples=120,
    ] {((3-x-2*y)^2 + (2-2*x-y)^2 + (6-3*x-4*y)^2)/3};

    % Evaluation point (0, 1)
    % residual₁ = 3-0-2 = 1      → ∇J₁ = -2(1)[1,2] = [-2, -4]
    % residual₂ = 2-0-1 = 1      → ∇J₂ = -2(1)[2,1] = [-4, -2]
    % residual₃ = 6-0-4 = 2      → ∇J₃ = -2(2)[3,4] = [-12, -16]
    % ∇J = 1/3[-2-4-12, -4-2-16] = [-6, -22/3]
    \addplot[only marks, mark=diamond*, mark options={scale=3, fill=hpurple}]
      coordinates {(0,1)};

    % Gradient arrows (scaled by 0.04)
    % ∇J₁ = [-2,-4] → [-0.08, -0.16]
    \draw[-{Stealth[length=6pt,width=4pt]}, line width=2pt, brown]
      (axis cs:0,1) -- (axis cs:-0.08,0.84);

    % ∇J₂ = [-4,-2] → [-0.16, -0.08]
    \draw[-{Stealth[length=6pt,width=4pt]}, line width=2pt, blue]
      (axis cs:0,1) -- (axis cs:-0.16,0.92);

    % ∇J₃ = [-12,-16] → [-0.48, -0.64]
    \draw[-{Stealth[length=6pt,width=4pt]}, line width=2pt, gray]
      (axis cs:0,1) -- (axis cs:-0.48,0.36);

    % ∇J = [-6, -22/3] → [-0.24, -0.293]
    \draw[-{Stealth[length=6pt,width=4pt]}, line width=2pt, green!80!black]
      (axis cs:0,1) -- (axis cs:-0.24,0.707);

    % Boxes on foreground layer (above grid)
    \begin{pgfonlayer}{axis foreground}
    % Gradient legend (top right)
    \node[anchor=south east, fill=white, draw=gray!50, inner sep=6pt, rounded corners=2pt]
      at (axis cs:1.45,0.05) {%
      \small
      \begin{tabular}{@{}cl@{}}
        \tikz[baseline=-0.3ex]\draw[-{Stealth[length=4pt,width=3pt]}, line width=2pt, brown] (0,0) -- (0.1,0); & $\nabla J_1$ \\
        \tikz[baseline=-0.3ex]\draw[-{Stealth[length=4pt,width=3pt]}, line width=2pt, blue] (0,0) -- (0.1,0); & $\nabla J_2$ \\
        \tikz[baseline=-0.3ex]\draw[-{Stealth[length=4pt,width=3pt]}, line width=2pt, gray] (0,0) -- (0.1,0); & $\nabla J_3$ \\
        \tikz[baseline=-0.3ex]\draw[-{Stealth[length=4pt,width=3pt]}, line width=2pt, green!80!black] (0,0) -- (0.1,0); & $\nabla J$ \\
      \end{tabular}%
    };

    \end{pgfonlayer}
  \end{axis}
\end{tikzpicture}
\end{document}
